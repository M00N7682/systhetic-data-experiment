\documentclass[11pt,a4paper]{article}

% 패키지
\usepackage[utf8]{inputenc}
\usepackage{kotex}  % 한글 지원
\usepackage{amsmath,amssymb,amsthm}
\usepackage{graphicx}
\usepackage{booktabs}
\usepackage{hyperref}
\usepackage{algorithm}
\usepackage{algorithmic}
\usepackage{geometry}
\usepackage{tikz}
\usetikzlibrary{shapes,arrows,positioning}
\geometry{margin=2.5cm}

% 정리 환경 정의
\newtheorem{theorem}{정리}
\newtheorem{lemma}{보조정리}
\newtheorem{definition}{정의}
\newtheorem{assumption}{가정}

% 제목
\title{Tail-Conditional KS Distance: \\
보험 합성 데이터의 조건부 극단 손실 보존 평가를 위한 일관된 통계적 지표}

\author{문덕룡 (Deok Lyong Moon) \\
경희대학교 경영학과 \\
\texttt{dfjk71@khu.ac.kr}}

\date{}

\begin{document}

\maketitle

%==============================================================================
\begin{abstract}
%==============================================================================

합성 데이터는 개인정보 보호와 데이터 활용 사이의 균형을 제공하는 유망한 대안으로 주목받고 있다. 그러나 기존 합성 데이터 품질 평가 지표는 전체 분포의 평균적 유사성에 초점을 맞추고 있어, 보험 리스크 관점에서 핵심적인 조건부 극단 손실 분포의 왜곡을 충분히 탐지하지 못하는 한계를 가진다.

본 연구는 Kolmogorov-Smirnov Distance를 조건부 분포 및 극단 손실 영역으로 확장한 \textbf{Tail-Conditional KS Distance (T-cKS)}를 제안하고, 이 지표의 통계적 일관성(consistency)을 이론적으로 증명한다.

Allstate Claims Severity 데이터셋을 활용한 통제된 실험 결과, 조건부 극단 손실이 80\% 축소된 상황에서 기존 지표(mKS, cKS)는 KS=0.017 수준으로 왜곡을 거의 탐지하지 못한 반면, 제안하는 T-cKS는 KS=0.344로 \textbf{약 20배 높은 민감도}를 보였다. 또한 실제 합성 데이터 생성 모델(CTGAN, TVAE, GaussianCopula)을 사용한 실험에서, TVAE가 기존 지표(cKS=0.021)로는 우수해 보이지만 T-cKS(0.075)가 \textbf{3.5배 높은 tail 왜곡}을 탐지함을 확인하였다.

\vspace{0.3cm}
\noindent\textbf{키워드}: 합성 데이터, 품질 평가, Kolmogorov-Smirnov Distance, 극단값, 보험 리스크, 조건부 분포, 통계적 일관성

\end{abstract}

%==============================================================================
\section{서론}
%==============================================================================

\subsection{연구 배경 및 필요성}

보험 산업에서는 개인정보 보호 규제 강화와 데이터 활용 제한으로 인해 실제 고객 데이터를 활용한 분석과 모델 개발에 구조적인 어려움이 지속되고 있다. 특히 보험 데이터는 개인의 건강 정보, 재무 정보, 사고 이력 등 민감한 정보를 포함하고 있어, 데이터 공유 및 외부 협업 과정에서 활용 가능성이 제한된다.

이러한 환경에서 합성 데이터(synthetic data)가 유망한 대안으로 주목받고 있다. 그러나 합성 데이터의 활용 가능성이 확대됨에 따라, \textbf{생성된 데이터가 실제 보험 리스크를 얼마나 충실히 반영하는지를 평가하는 문제}의 중요성이 증가하고 있다.

\subsection{연구 문제 정의}

기존 합성 데이터 품질 평가는 주로 Kolmogorov-Smirnov(KS) Distance와 같은 통계적 거리 지표를 활용한다. 그러나 보험 데이터의 손실 분포는 일반적으로 heavy-tailed 특성을 가지며, 전체 손실의 상당 부분이 소수의 고액 손실에 의해 결정된다.

이에 본 연구는 다음과 같은 연구 질문을 제기한다:

\begin{quote}
\textit{``합성 데이터 평가 지표는 보험 리스크 관점에서 중요한 조건부 극단 손실 분포의 왜곡을 효과적으로 탐지할 수 있는가?"}
\end{quote}

\subsection{연구 기여점}

본 연구의 기여점은 다음과 같다:

\begin{enumerate}
    \item 보험 합성 데이터 평가 문제를 조건부 극단 손실 보존이라는 리스크 중심 관점에서 재정의
    \item 새로운 평가 지표 T-cKS를 제안하고, 통계적 일관성을 이론적으로 증명
    \item 통제된 실험에서 제안 지표의 탐지 능력이 기존 지표 대비 약 20배 높음을 확인
    \item 실제 합성 데이터 생성 모델(CTGAN, TVAE, GaussianCopula)에서도 T-cKS의 유용성을 실증
\end{enumerate}


%==============================================================================
\section{관련 연구}
%==============================================================================

\subsection{합성 데이터 생성 기술}

합성 데이터 생성 기술은 크게 통계적 방법과 딥러닝 기반 방법으로 구분된다. 통계적 방법으로는 Gaussian Copula~[6]가 대표적이며, 변수 간 의존성을 코퓰라 함수로 모델링한다. 딥러닝 기반 방법으로는 CTGAN~[4]이 조건부 GAN 구조를 활용하여 범주형 변수가 포함된 테이블 데이터를 생성하고, TVAE~[5]는 Variational Autoencoder를 테이블 데이터에 적용한다.

이러한 생성 모델들은 전체 분포의 유사성을 목표로 학습되므로, 데이터의 특정 영역(예: tail)에서 품질 저하가 발생할 수 있다.

\subsection{합성 데이터 품질 평가}

합성 데이터 품질 평가를 위한 다양한 지표가 제안되어 왔다. 주변 분포 유사성 측정을 위해 Kolmogorov-Smirnov (KS) Distance, Wasserstein Distance, Maximum Mean Discrepancy (MMD) 등이 활용된다~[1,2]. SDMetrics~[6]는 이러한 지표들을 통합한 평가 프레임워크를 제공한다.

그러나 기존 지표들은 다음과 같은 한계를 가진다:
\begin{itemize}
    \item 전체 분포를 단일 값으로 요약하여 국소적 차이를 감지하기 어려움
    \item 조건부 분포의 품질을 직접 평가하지 않음
    \item 극단값 영역에 대한 가중치가 없어 tail 왜곡에 둔감함
\end{itemize}

\subsection{극단값 이론과 보험 리스크}

보험 분야에서는 극단값 이론(Extreme Value Theory, EVT)을 활용한 리스크 분석이 활발히 이루어져 왔다~[3]. Value-at-Risk (VaR)와 Conditional Value-at-Risk (CVaR) 등의 리스크 측정 지표는 손실 분포의 tail 특성에 크게 의존한다.

그러나 합성 데이터 품질 평가 지표와 극단값 이론을 직접적으로 결합한 연구는 상대적으로 부족하다. 본 연구는 이 gap을 메우기 위해 tail 영역에 특화된 평가 지표를 제안한다.


%==============================================================================
\section{제안 방법: Tail-Conditional KS Distance}
%==============================================================================

\subsection{문제 정의}

다음과 같은 데이터 구조를 가정한다:
\begin{itemize}
    \item 실제 데이터: $\{(y_i, z_i)\}_{i=1}^n$, $y_i$는 손실액, $z_i$는 조건 변수
    \item 합성 데이터: $\{(\tilde{y}_j, \tilde{z}_j)\}_{j=1}^m$
    \item $Y$: 연속형 손실 변수 (heavy-tailed 분포)
    \item $Z$: 범주형 조건 변수 (예: 위험군 분류)
\end{itemize}

\subsection{기존 평가 지표}

\begin{definition}[Marginal KS Distance]
두 분포의 누적분포함수(CDF) 간 최대 차이:
\begin{equation}
    \text{mKS} = \sup_y |F_{\text{real}}(y) - F_{\text{synth}}(y)|
\end{equation}
\end{definition}

\begin{definition}[Conditional KS Distance]
각 조건 $z$에서 KS Distance를 계산하고 가중 평균:
\begin{equation}
    \text{cKS} = \sum_{z} w_z \cdot \text{KS}(Y_{\text{real}}|Z=z, Y_{\text{synth}}|Z=z)
\end{equation}
여기서 $w_z$는 조건 $z$의 샘플 비율이다.
\end{definition}

\subsubsection{지표 비교 요약}

표~\ref{tab:metric_comparison}은 세 가지 평가 지표의 특성을 비교한다.

\begin{table}[h]
\centering
\caption{mKS, cKS, T-cKS 비교 요약}
\begin{tabular}{@{}lccc@{}}
\toprule
특성 & mKS & cKS & \textbf{T-cKS (제안)} \\
\midrule
평가 범위 & 전체 분포 & 조건별 분포 & 조건별 Tail 분포 \\
조건 변수 고려 & $\times$ & $\checkmark$ & $\checkmark$ \\
Tail 영역 집중 & $\times$ & $\times$ & $\checkmark$ \\
\midrule
\multicolumn{4}{l}{\textit{탐지 능력 (본 연구 실험 결과)}} \\
\midrule
80\% Tail 축소 시 & 0.017 & 0.017 & \textbf{0.344} \\
탐지 민감도 & 1x & 1x & \textbf{20x} \\
\midrule
\multicolumn{4}{l}{\textit{실제 합성 모델 평가 (TVAE, $q=0.95$)}} \\
\midrule
지표 값 & 0.014 & 0.021 & \textbf{0.075} \\
숨겨진 Tail 왜곡 탐지 & $\times$ & $\times$ & $\checkmark$ \\
\bottomrule
\end{tabular}
\label{tab:metric_comparison}
\end{table}

mKS와 cKS는 전체 분포 또는 조건별 분포의 유사성을 측정하지만, tail 영역의 왜곡에는 둔감하다. 반면 T-cKS는 조건별 tail 분포에 집중하여 보험 리스크 관점에서 중요한 극단 손실 왜곡을 효과적으로 탐지한다.

\subsection{T-cKS 정의}

\begin{definition}[Tail-Conditional KS Distance]
T-cKS는 다음 4단계로 계산된다:

\textbf{Step 1: Tail Threshold 정의}
\begin{equation}
    \tau_q = Q_q(Y_{\text{real}})
\end{equation}
여기서 $Q_q$는 $q$-분위수 함수이다 (예: $q = 0.95$).

\textbf{Step 2: Tail Subset 필터링}
\begin{equation}
    Y_{\text{tail}}^z = \{y \in Y \mid Z = z \land y > \tau_q\}
\end{equation}

\textbf{Step 3: Tail-Conditional KS 계산}
\begin{equation}
    \text{T-KS}_z = \sup_{y > \tau_q} |\hat{F}_z^{\text{tail}}(y) - \hat{G}_z^{\text{tail}}(y)|
\end{equation}

\textbf{Step 4: 가중 평균}
\begin{equation}
    \text{T-cKS} = \sum_{z} \frac{n_z^{\text{tail}}}{\sum_z n_z^{\text{tail}}} \cdot \text{T-KS}_z
\end{equation}
\end{definition}

% 그림 1: T-cKS 계산 플로우차트
\begin{figure}[h]
    \centering
    \begin{tikzpicture}[
        node distance=1.2cm,
        box/.style={rectangle, draw, rounded corners, minimum width=2.5cm, minimum height=0.8cm, align=center, font=\small},
        arrow/.style={->, thick}
    ]
    % 노드 정의
    \node[box, fill=blue!20] (data) {전체 데이터\\$(Y, Z)$};
    \node[box, fill=green!20, right=of data] (cond) {조건별 분리\\$Y|Z=z$};
    \node[box, fill=orange!20, right=of cond] (tail) {Tail 필터링\\$Y > \tau_q$};
    \node[box, fill=red!20, right=of tail] (ks) {KS 계산\\$\text{T-KS}_z$};
    \node[box, fill=purple!20, below=0.8cm of ks] (avg) {가중 평균\\$\text{T-cKS}$};
    
    % 화살표
    \draw[arrow] (data) -- (cond);
    \draw[arrow] (cond) -- (tail);
    \draw[arrow] (tail) -- (ks);
    \draw[arrow] (ks) -- (avg);
    
    % 라벨
    \node[above=0.1cm of data, font=\footnotesize] {Step 0};
    \node[above=0.1cm of cond, font=\footnotesize] {Step 1};
    \node[above=0.1cm of tail, font=\footnotesize] {Step 2};
    \node[above=0.1cm of ks, font=\footnotesize] {Step 3};
    \node[left=0.1cm of avg, font=\footnotesize] {Step 4};
    \end{tikzpicture}
    \caption{T-cKS 계산 과정: 전체 데이터에서 조건별 분리, tail 영역 필터링, 각 조건에서 KS Distance 계산 후 가중 평균}
    \label{fig:flowchart}
\end{figure}

\subsection{이론적 성질: 일관성 (Consistency)}

\begin{assumption}
다음 조건들을 가정한다:
\begin{enumerate}
    \item[(A1)] $Y$는 연속 분포 $F$를 따르며, 밀도함수 $f$가 존재한다.
    \item[(A2)] 각 조건 $z$에서 $P(Z=z) > 0$이고, $P(Y > \tau_q | Z=z) > 0$이다.
    \item[(A3)] $\tau_q$는 실제 데이터의 $q$-분위수로 고정된 값이다.
    \item[(A4)] 실제 데이터와 합성 데이터의 샘플은 각각 독립적으로 추출된다.
\end{enumerate}
\end{assumption}

\begin{theorem}[일관성]
가정 (A1)-(A4) 하에서, $n, m \to \infty$이고 각 조건의 tail 샘플 수 $n_z^{\text{tail}}, m_z^{\text{tail}} \to \infty$일 때, 다음이 성립한다:
\begin{equation}
    \text{T-cKS}(\hat{F}_n, \hat{G}_m) \xrightarrow{p} D_\tau(F, G)
\end{equation}
여기서:
\begin{itemize}
    \item $\xrightarrow{p}$는 확률 수렴(convergence in probability)
    \item $D_\tau(F, G) = \sum_z w_z^* \cdot \sup_{y>\tau_q} |F_z^{\text{tail}}(y) - G_z^{\text{tail}}(y)|$는 참 분포 간의 tail-conditional 거리
\end{itemize}
\end{theorem}

\begin{proof}[증명]

\textbf{Step 1: 각 조건 $z$에서 경험적 tail CDF의 균등 수렴}

Glivenko-Cantelli 정리를 tail 영역에 적용하면, 가정 (A1), (A2) 하에서:
\begin{equation}
    \sup_{y>\tau_q} |\hat{F}_z^{\text{tail}}(y) - F_z^{\text{tail}}(y)| \xrightarrow{p} 0 \quad \text{as } n_z^{\text{tail}} \to \infty
\end{equation}

\textbf{Step 2: Continuous Mapping Theorem 적용}

함수 $g(F, G) = \sup_y |F(y) - G(y)|$는 균등 수렴 위상에서 연속이다. 따라서:
\begin{equation}
    \text{T-KS}_z = g(\hat{F}_z^{\text{tail}}, \hat{G}_z^{\text{tail}}) \xrightarrow{p} g(F_z^{\text{tail}}, G_z^{\text{tail}}) = D_z
\end{equation}

\textbf{Step 3: 가중 평균의 수렴}

각 조건의 가중치 $w_z = n_z^{\text{tail}} / \sum_z n_z^{\text{tail}}$는 참 가중치 $w_z^* = P(Y>\tau_q, Z=z) / P(Y>\tau_q)$로 수렴한다.

각 T-KS$_z$가 $D_z$로 확률 수렴하고, 가중치도 수렴하므로:
\begin{equation}
    \text{T-cKS} = \sum_z w_z \cdot \text{T-KS}_z \xrightarrow{p} \sum_z w_z^* \cdot D_z = D_\tau(F, G)
\end{equation}
\end{proof}

\textbf{정리 1의 의미:}
\begin{itemize}
    \item 샘플 크기가 증가하면 T-cKS는 참 값으로 수렴한다.
    \item $F = G$이면 $D_\tau(F, G) = 0$이므로 T-cKS $\xrightarrow{p} 0$
    \item $F \neq G$이고 tail에서 차이가 있으면 T-cKS가 이를 탐지한다.
\end{itemize}


%==============================================================================
\section{실험}
%==============================================================================

\subsection{데이터셋}

Allstate Claims Severity Dataset (OpenML)을 사용하였다~[7].

\begin{table}[h]
\centering
\caption{데이터셋 요약}
\begin{tabular}{@{}ll@{}}
\toprule
항목 & 값 \\
\midrule
총 샘플 수 & 188,318 \\
손실 변수 ($Y$) & loss (연속형) \\
조건 변수 ($Z$) & cat79 (4개 범주: A, B, C, D) \\
\bottomrule
\end{tabular}
\end{table}

\begin{table}[h]
\centering
\caption{손실 분포 통계량}
\begin{tabular}{@{}lr@{}}
\toprule
통계량 & 값 \\
\midrule
평균 & 3,037.34 \\
중앙값 & 2,115.57 \\
표준편차 & 2,904.09 \\
왜도 (Skewness) & 3.79 \\
첨도 (Kurtosis) & 48.08 \\
95\% 분위수 ($\tau_{0.95}$) & 8,508.54 \\
최대값 & 121,012.25 \\
\bottomrule
\end{tabular}
\end{table}

왜도 $> 1$, 첨도 $> 3$으로 heavy-tailed 분포임을 확인하였다.

% 그림 2
\begin{figure}[h]
    \centering
    \includegraphics[width=0.9\textwidth]{figures/q_sensitivity_winsor.png}
    \caption{Quantile 민감도 분석 (Winsorization, strength=0.8)}
    \label{fig:loss_dist}
\end{figure}

\subsection{실패 모드 설계: Tail Scaling}

실제 합성 데이터에서 발생할 수 있는 tail 왜곡을 시뮬레이션하기 위해, \textbf{Tail Scaling} 방법을 설계하였다. 이 방법은 특정 조건(본 실험에서는 조건 D)의 tail 값만 선택적으로 축소하여, 조건부 극단 손실 분포의 왜곡을 유도한다.

조건 D의 tail 값($Y > \tau_q$)을 다음과 같이 스케일링한다:
\begin{equation}
    Y_{\text{new}} = \tau_q + (Y - \tau_q) \times \text{scale\_factor}
\end{equation}

이 변환은 다음과 같은 특성을 가진다:
\begin{itemize}
    \item tail threshold $\tau_q$ 이하의 값은 변경되지 않음
    \item tail 영역의 값만 threshold 방향으로 압축됨
    \item scale\_factor가 작을수록 더 심한 왜곡 발생
\end{itemize}

실험에서 사용한 scale\_factor 값:
\begin{itemize}
    \item scale\_factor $= 1.0$: 원본 (왜곡 없음)
    \item scale\_factor $= 0.8$: tail의 20\% 축소
    \item scale\_factor $= 0.6$: tail의 40\% 축소
    \item scale\_factor $= 0.4$: tail의 60\% 축소
    \item scale\_factor $= 0.2$: tail의 80\% 축소 (가장 극심한 왜곡)
\end{itemize}

이러한 왜곡은 전체 데이터의 약 5\% (tail 영역)에만 영향을 미치므로, 전체 분포 관점의 지표(mKS, cKS)로는 탐지가 어렵다.

% 그림 3
\begin{figure}[h]
    \centering
    \includegraphics[width=0.9\textwidth]{figures/metric_comparison_q0.9.png}
    \caption{메트릭 비교 (q=0.90)}
    \label{fig:tail_scaling}
\end{figure}


%==============================================================================
\section{실험 결과}
%==============================================================================

\subsection{주요 결과}

\begin{table}[h]
\centering
\caption{Tail Scaling 강도별 평가 지표 비교 ($q=0.95$, 왜곡 대상: 조건 D)}
\begin{tabular}{@{}lcccc@{}}
\toprule
Scale Factor & mKS & cKS & T-cKS & T-cKS(D) \\
\midrule
1.0 (원본) & 0.0000 & 0.0000 & 0.0001 & 0.0002 \\
0.8 (20\% 축소) & 0.0028 & 0.0028 & 0.0562 & 0.0861 \\
0.6 (40\% 축소) & 0.0061 & 0.0061 & 0.1221 & 0.1873 \\
0.4 (60\% 축소) & 0.0106 & 0.0106 & 0.2120 & 0.3250 \\
\textbf{0.2 (80\% 축소)} & \textbf{0.0172} & \textbf{0.0172} & \textbf{0.3441} & \textbf{0.5276} \\
\bottomrule
\end{tabular}
\label{tab:main_results}
\end{table}

% 그림 4
\begin{figure}[h]
    \centering
    \includegraphics[width=0.9\textwidth]{figures/metric_comparison_q0.95.png}
    \caption{왜곡 강도에 따른 평가 지표 변화 (q=0.95)}
    \label{fig:main_plot}
\end{figure}

\subsection{결과 분석}

조건 D의 극단 손실이 80\% 축소된 상황(scale=0.2)에서:

\textbf{기존 지표의 한계:}
\begin{itemize}
    \item mKS $= 0.0172$: 전체 분포 관점에서는 거의 차이 없음
    \item cKS $= 0.0172$: 조건별로 분리해도 여전히 낮음
\end{itemize}

\textbf{T-cKS의 탐지 능력:}
\begin{itemize}
    \item T-cKS $= 0.3441$: 유의미한 차이 탐지
    \item T-cKS(D) $= 0.5276$: 왜곡 조건만 보면 극명한 차이
\end{itemize}

\textbf{탐지력 비교:}
\begin{equation}
    \frac{\text{T-cKS}}{\text{cKS}} = \frac{0.3441}{0.0172} \approx \textbf{20배}
\end{equation}

\subsection{조건별 상세 분석}

\begin{table}[h]
\centering
\caption{조건별 KS Distance 상세 (scale=0.2)}
\begin{tabular}{@{}lcc@{}}
\toprule
조건 & Conditional KS & Tail-Conditional KS \\
\midrule
A & 0.0000 & 0.0000 \\
B & 0.0000 & 0.0000 \\
C & 0.0000 & 0.0000 \\
\textbf{D (왜곡)} & \textbf{0.1843} & \textbf{0.5276} \\
\midrule
가중 평균 & 0.0172 & 0.3441 \\
\bottomrule
\end{tabular}
\end{table}

% 그림 5
\begin{figure}[h]
    \centering
    \includegraphics[width=0.9\textwidth]{figures/detection_comparison_q095.png}
    \caption{탐지 능력 비교: T-cKS vs 기존 지표 (q=0.95)}
    \label{fig:condition_compare}
\end{figure}

\subsection{일관성 검증}

\begin{table}[h]
\centering
\caption{샘플 크기에 따른 T-cKS 추정의 표준편차}
\begin{tabular}{@{}lc@{}}
\toprule
샘플 비율 & T-cKS 표준편차 \\
\midrule
10\% & 0.0521 \\
25\% & 0.0298 \\
50\% & 0.0187 \\
100\% & 0.0123 \\
\bottomrule
\end{tabular}
\end{table}

샘플 크기가 증가함에 따라 표준편차가 감소하여, 정리 1의 일관성이 실험적으로 확인되었다.

% 그림 6
\begin{figure}[h]
    \centering
    \includegraphics[width=0.9\textwidth]{figures/heatmap_tcks_winsorization.png}
    \caption{T-cKS 히트맵: Quantile vs 왜곡 강도 (Winsorization 방법)}
    \label{fig:consistency}
\end{figure}


\subsection{실제 합성 데이터 생성 모델 평가}

앞선 실험은 인위적 왜곡을 통한 통제된 환경에서 T-cKS의 민감도를 검증하였다. 본 절에서는 실제 합성 데이터 생성 모델을 사용하여 T-cKS의 실용적 가치를 평가한다.

\subsubsection{실험 설정}

SDV(Synthetic Data Vault) 라이브러리의 세 가지 합성 데이터 생성 모델을 평가하였다:
\begin{itemize}
    \item \textbf{GaussianCopula}: 통계적 방법 기반 모델
    \item \textbf{CTGAN}: Conditional GAN 기반 딥러닝 모델
    \item \textbf{TVAE}: Variational Autoencoder 기반 딥러닝 모델
\end{itemize}

각 모델에서 50,000개의 합성 샘플을 생성하고, 3개의 랜덤 시드(42, 123, 456)로 반복 실험을 수행하였다.

\subsubsection{주요 결과}

\begin{table}[h]
\centering
\caption{실제 합성 데이터 생성 모델 평가 결과 ($q=0.95$)}
\begin{tabular}{@{}lccccc@{}}
\toprule
모델 & mKS & cKS & T-cKS & T-cKS/cKS & Tail 재현율 \\
\midrule
\textbf{TVAE} & \textbf{0.014} & \textbf{0.021} & 0.075 & \textbf{3.51x} & 25.4\% \\
CTGAN & 0.086 & 0.072 & 0.125 & 1.74x & 43.2\% \\
GaussianCopula & 0.068 & 0.141 & 0.142 & 1.00x & 17.7\% \\
\bottomrule
\end{tabular}
\label{tab:synthetic_results}
\end{table}

% 그림 7: 메트릭 비교
\begin{figure}[h]
    \centering
    \includegraphics[width=0.9\textwidth]{figures/synthetic_metric_comparison_q095.png}
    \caption{실제 합성 데이터 생성 모델별 품질 지표 비교 ($q=0.95$)}
    \label{fig:synthetic_comparison}
\end{figure}

\subsubsection{핵심 발견: TVAE의 숨겨진 Tail 왜곡}

TVAE는 기존 지표에서 가장 우수한 성능을 보인다:
\begin{itemize}
    \item mKS $= 0.014$: 전체 분포 유사성 최고
    \item cKS $= 0.021$: 조건부 분포 유사성 최고
\end{itemize}

그러나 T-cKS는 숨겨진 tail 왜곡을 탐지한다:
\begin{equation}
    \frac{\text{T-cKS}}{\text{cKS}} = \frac{0.075}{0.021} \approx \textbf{3.5배}
\end{equation}

실제로 TVAE는 원본 대비 \textbf{25.4\%}의 tail 샘플만 생성하여, 극단 손실 영역의 재현에 실패하였다.

% 그림 8: TVAE 상세 분석
\begin{figure}[h]
    \centering
    \includegraphics[width=0.8\textwidth]{figures/tvae_hidden_distortion.png}
    \caption{TVAE: 기존 지표로는 우수하나 T-cKS가 숨겨진 tail 왜곡 탐지}
    \label{fig:tvae_detail}
\end{figure}

% 그림 9: 탐지 민감도
\begin{figure}[h]
    \centering
    \includegraphics[width=0.8\textwidth]{figures/synthetic_tcks_sensitivity.png}
    \caption{Tail 탐지 민감도: T-cKS / cKS 비율}
    \label{fig:sensitivity}
\end{figure}

\subsubsection{Tail 재현율 분석}

% 그림 10: Tail 재현율
\begin{figure}[h]
    \centering
    \includegraphics[width=0.95\textwidth]{figures/synthetic_tail_reproduction.png}
    \caption{합성 데이터 생성 모델별 Tail 재현율: (a) Tail 샘플 생성 비율, (b) Tail 평균 보존율}
    \label{fig:tail_reproduction}
\end{figure}

모든 모델이 tail 샘플 생성에 어려움을 겪고 있다:
\begin{itemize}
    \item GaussianCopula: 17.7\% (가장 낮음)
    \item TVAE: 25.4\%
    \item CTGAN: 43.2\% (가장 높음)
\end{itemize}

이는 현재 합성 데이터 생성 기술이 heavy-tailed 분포의 극단 영역 재현에 근본적 한계를 가짐을 시사한다.

% 그림 11: Quantile별 T-cKS
\begin{figure}[h]
    \centering
    \includegraphics[width=0.85\textwidth]{figures/synthetic_tcks_by_quantile.png}
    \caption{Tail Threshold Quantile에 따른 T-cKS 변화}
    \label{fig:tcks_quantile}
\end{figure}

\subsubsection{실험의 의의}

본 실험 결과는 T-cKS의 실용적 가치를 입증한다:
\begin{enumerate}
    \item \textbf{숨겨진 문제 탐지}: 기존 지표로는 ``우수"해 보이는 TVAE의 tail 왜곡을 탐지
    \item \textbf{모델 선택 가이드}: 보험 리스크 관점에서 합성 데이터 생성 모델 평가 기준 제공
    \item \textbf{생성 모델의 한계 규명}: 현재 합성 데이터 기술의 heavy-tail 재현 한계 확인
\end{enumerate}


%==============================================================================
\section{논의}
%==============================================================================

\subsection{연구 의의}

본 연구는 보험 합성 데이터 평가 문제를 리스크 중심 관점에서 재정의하였다. 기존 평가 방식의 근본적 한계는 ``평균적으로 유사하면 좋은 합성 데이터"라는 암묵적 가정에 있다. 그러나 보험 리스크 관리에서 중요한 것은 평균이 아닌 \textbf{극단 손실}이다.

\subsubsection{이론적 기여}
정리 1을 통해 T-cKS의 통계적 일관성을 이론적으로 보장하였다. 이는 샘플 크기가 충분히 크면 T-cKS가 참 분포 간의 tail-conditional 거리로 수렴함을 의미하며, 실험적으로도 샘플 크기 증가에 따른 표준편차 감소(표 5)를 통해 확인하였다.

\subsubsection{실용적 기여}
TVAE 실험 결과는 T-cKS의 실용적 가치를 명확히 보여준다. TVAE는 mKS=0.014, cKS=0.021로 기존 지표에서 가장 우수한 성능을 보였으나, T-cKS=0.075로 3.5배 높은 tail 왜곡이 탐지되었다. 이는 기존 평가 방식만으로는 ``좋아 보이지만 실제로는 리스크 관점에서 문제가 있는" 합성 데이터를 걸러내기 어려움을 시사한다.

\subsubsection{합성 데이터 생성 기술에 대한 시사점}
실험 결과, 현재 최신 합성 데이터 생성 모델(CTGAN, TVAE, GaussianCopula) 모두 tail 재현율이 17--43\%에 불과하였다. 이는 heavy-tailed 분포의 극단 영역 재현이 현재 생성 기술의 근본적 한계임을 보여주며, 향후 tail-aware 학습 방법 개발의 필요성을 제기한다.

\subsection{한계점 및 향후 연구}

\begin{enumerate}
    \item \textbf{단일 데이터셋}: 다른 도메인(의료, 금융 등)에서의 일반화 검증 필요
    \item \textbf{합성 모델 다양성}: 더 다양한 합성 데이터 생성 모델(Diffusion, Normalizing Flow 등) 평가 필요
    \item \textbf{점근적 분포}: T-cKS의 유의성 검정 방법 개발 필요
    \item \textbf{다변량 확장}: 다변량 극단 의존성 평가로의 확장 가능
    \item \textbf{Tail 재현 개선}: 합성 데이터 생성 모델의 tail 재현율 향상 방법 연구
\end{enumerate}


%==============================================================================
\section{결론}
%==============================================================================

본 연구는 보험 합성 데이터 평가에서 기존 분포 기반 지표의 한계를 분석하고, 조건부 극단 손실 보존을 평가하는 \textbf{Tail-Conditional KS Distance (T-cKS)}를 제안하였다. 또한 T-cKS의 통계적 일관성을 이론적으로 증명하였다.

실험 결과:
\begin{itemize}
    \item \textbf{통제된 실험}: 기존 지표(mKS, cKS)는 KS $= 0.017$ 수준으로 왜곡 탐지 실패, T-cKS는 KS $= 0.344$로 \textbf{약 20배 높은 민감도}
    \item \textbf{실제 합성 모델}: TVAE가 기존 지표(cKS$=0.021$)로는 우수하나, T-cKS($=0.075$)가 \textbf{3.5배 높은 tail 왜곡} 탐지
    \item \textbf{생성 모델 한계}: 현재 합성 데이터 기술은 tail 샘플의 17--43\%만 재현 가능
\end{itemize}

이는 T-cKS가 보험 합성 데이터의 리스크 관점 품질 평가에 효과적이며, 기존 지표로는 발견하기 어려운 숨겨진 tail 왜곡을 탐지할 수 있음을 시사한다.


%==============================================================================
\section*{참고문헌}
%==============================================================================

\begin{enumerate}
    \item Kolmogorov, A. N. (1933). Sulla determinazione empirica di una legge di distribuzione. \textit{Giornale dell'Istituto Italiano degli Attuari}, 4, 83-91.
    
    \item Smirnov, N. (1948). Table for estimating the goodness of fit of empirical distributions. \textit{Annals of Mathematical Statistics}, 19(2), 279-281.
    
    \item Embrechts, P., Klüppelberg, C., \& Mikosch, T. (1997). \textit{Modelling Extremal Events for Insurance and Finance}. Springer.
    
    \item Xu, L., et al. (2019). Modeling tabular data using conditional GAN. \textit{NeurIPS}, 32.
    
    \item Kingma, D. P., \& Welling, M. (2014). Auto-encoding variational bayes. \textit{ICLR}.
    
    \item Patki, N., Wedge, R., \& Veeramachaneni, K. (2016). The Synthetic Data Vault. \textit{IEEE DSAA}, 399-410.
    
    \item OpenML. (2016). Allstate Claims Severity Dataset. \url{https://www.openml.org/search?type=data&id=42571}
\end{enumerate}


%==============================================================================
\appendix
\section{증명 세부사항}
%==============================================================================

\begin{lemma}[Tail 영역에서의 Glivenko-Cantelli]
$\{Y_i\}$가 연속 분포 $F$를 따르는 i.i.d. 샘플이고, $\tau$가 고정된 threshold일 때, tail 경험적 CDF에 대해:
\begin{equation}
    \sup_{y > \tau} |\hat{F}_\tau(y) - F_\tau(y)| \xrightarrow{a.s.} 0
\end{equation}
여기서 $F_\tau(y) = P(Y \leq y | Y > \tau)$이다.
\end{lemma}

\begin{proof}
조건부 분포에 대한 Glivenko-Cantelli 정리의 직접 적용.
\end{proof}

\begin{lemma}[KS 거리의 연속성]
$g: (F, G) \mapsto \sup_y |F(y) - G(y)|$는 균등 수렴 위상에서 연속이다.
\end{lemma}

\begin{proof}
\begin{equation}
    |g(F_n, G_n) - g(F, G)| \leq \sup_y |F_n(y) - F(y)| + \sup_y |G_n(y) - G(y)|
\end{equation}
따라서 $F_n \to F$, $G_n \to G$이면 $g(F_n, G_n) \to g(F, G)$.
\end{proof}


%==============================================================================
\section{재현 코드}
%==============================================================================

실험 코드는 다음 저장소에서 확인할 수 있다:

\url{https://github.com/M00N7682/systhetic-data-experiment}


\end{document}

